%%%%%%%%%%%%%%%%%%%%%%%%%%%%%%%%%%%%%%%%%%%%%%%
% INTRODUCCION   %
%%%%%%%%%%%%%%%%%%%%%%%%%%%%%%%%%%%%%%%%%%%%%%%
\chapter{Introducción y antecedentes}
Explicar los conceptos, no necesariamente en este orden, de:
\begin{itemize}
	\item Microfísica de nubes
	\item Lluvia caliente
	\item Procesos de nucleación / Coalesencia
	\item Historia y antecedentes de mediciones
	\item Espectros 
	\item Modelos predictivos
	\item Regiones de medicion 
	\item Hipótesis central
\end{itemize}
%%%%%%%%%%%%%%%%%%%%%%%%%%%%%%%%%%%%%%%%%%%%%%%%%%%%%%%%%%%%%%%%%%%%%%%%%
\section{Microfísica de nubes} 
La lluvia es uno de los fenómenos meteorológicos mas importantes y más comunes en la naturaleza. Ésta es fundamental en el ciclo del agua y en muchas otras actividades de origen humano que van desde la agricultura, hasta la prevención de desastres naturales. 
Todas esas actividades dependen finalmente de tomar buenas medidas, hacer buenas estimaciones o  buenas predicciones de diferentes variables, dependiendo de que se quiera hacer, como pueden ser la precipitacion acumulada, intensidad, tiempo de duracion de la lluvia, etc...
Por lo tanto es necesario mejorar nuestro entendimiento acerca de estos fenómenos,


%%%%%%%%%%%%%%%%%%%%%%%%%%%%%%%%%%%%%%%%%%%%%%%%%%%%%%%%%%%%%%%%%%%%%%%%%
%                           Objetivo                                    %
%%%%%%%%%%%%%%%%%%%%%%%%%%%%%%%%%%%%%%%%%%%%%%%%%%%%%%%%%%%%%%%%%%%%%%%%%
\section{Introducción a procesos de lluvia caliente}
EEste trabajo tiene por objetivo ...Este trabajo tiene por objetivo ...Este trabajo tiene por objetivo ...Este trabajo tiene por objetivo ...ste trabajo tiene por objetivo ...
Este trabajo tiene por objetivo ...Este trabajo tiene por objetivo ...Este trabajo tiene por objetivo ...Este trabajo tiene por objetivo ...
Este trabajo tiene por objetivo ...vEste trabajo tiene por objetivo ...Este trabajo tiene por objetivo ...Este trabajo tiene por objetivo ...Este trabajo tiene por objetivo ...
Este trabajo tiene por objetivo ...Este trabajo tiene por objetivo ...Este trabajo tiene por objetivo ...
%%%%%%%%%%%%%%%%%%%%%%%%%%%%%%%%%%%%%%%%%%%%%%%%%%%%%%%%%%%%%%%%%%%%%%%%%
%                           Motivación y estado del arte                %
%%%%%%%%%%%%%%%%%%%%%%%%%%%%%%%%%%%%%%%%%%%%%%%%%%%%%%%%%%%%%%%%%%%%%%%%%
\section{Planteamiento de hipótesis}
Planteamiento de la hipotesis a validar/refutar