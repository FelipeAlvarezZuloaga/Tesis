\documentclass{book}																				% Clase para la tesis

\usepackage{etoolbox}
\patchcmd{\chapter}{\thispagestyle{plain}}{\thispagestyle{fancy}}{}{} % Para que en los nuevos capítulos, se incluya el encabezado y pie de página
\makeatletter

\usepackage[spanish]{babel}																	   % Para usar Español
\usepackage{amssymb, amsmath, amsbsy, amsfonts, mathptmx}  % Para poder usar mates
\usepackage{graphicx}																				% Para insertar gráficas, imágenes o figuras
\usepackage[table,xcdraw]{xcolor} 														 % Para el color en tablas
\usepackage{listings}                    															   % Para agregar codigo
\usepackage[T1]{fontenc}																		% Evita tener problemas con las fuentes italicas, negritas. NO FUNCIONA CON PDFLATEX

\usepackage{vmargin}																			    % Para modificar márgenes
\usepackage[raggedleft]{titlesec}														    % Alinea a la derecha los títulos de capitulos, secciones, etc...
\setpapersize{A4}																						 % Tamaño de papel
\setmargins{2.5cm}      																			  % Margen izquierdo
{1.5cm}                        																				   	  % Margen superior
{16.5cm}                      																					 % Anchura del texto
{23.42cm}                    																				   % Altura del texto
{10pt}                           																					% Altura de los encabezados
{1cm}                          																						% Espacio entre el texto y los encabezados
{0pt}                             																					  % Altura del pie de página
{2cm}                           																					% Espacio entre el texto y el pie de página


\usepackage{fancyhdr}																				% Para cambiar los encabezados y pies de página
\pagestyle{fancy}
\fancyhead[RO,LE]{\thesection}
\fancyfoot{} 
\fancyfoot[LE,RO]{}
\fancyfoot[LO,CE]{}
\fancyfoot[LO,RE]{\thepage}
\setlength{\headheight}{14pt}
\renewcommand{\headrulewidth}{1pt}
\renewcommand{\footrulewidth}{0.5pt}

\addto{\captionsspanish}{\renewcommand*{\contentsname}{Índice}}		% Cambia el nombre de "Índice General" a "Índice"
\renewcommand\bibname{Referencias}																	% Cambia el nombre de "Índice General" a "Referencias"

\begin{document}
%%%%%%%%%%%%%%%%%%%%%%%%%%%%%%%%%%%%%%%%%%%%%%%
\thispagestyle{empty}														% Fuerza a la siguiente página a quedar en blanco
\begin{minipage}{.2\textwidth}												% Se usa para poder "encimar" cosas sin problemas
  \flushleft																% Manda el contenido a la izquierda
  \center{\includegraphics[scale=.09]{Portada/unam.pdf}}					% Inserta el escudo de la UNAM

  \vspace{20pt}																% Hace un salto vertical

  \center{																	% Inserta las líneas
    \rule{.5pt}{.6\textheight}
    \hspace{7pt}
    \rule{2pt}{.6\textheight}
    \hspace{7pt}
    \rule{.5pt}{.6\textheight}
  } \\

  \center{\includegraphics[scale=.22]{Portada/ciencias.pdf}}				% Inserta el escuddo de Ciencias
\end{minipage}
\begin{minipage}{.7\textwidth}
	\flushright																% Manda el contenido a la derecha
	
	\center{
	
	  \center{
	    \LARGE{U}\large{NIVERSIDAD} \LARGE{N}\large{ACIONAL} 				% Crea el nombre estilizado de la universidad
	    \LARGE{A}\large{UTÓNOMA} \\[10pt]
	    \large{DE} 
	    \LARGE{M}\large{ÉXICO} 
	  } \\
	  \rule{\textwidth}{2pt}
	  \\
	  \hrulefill\\[1cm]
	  
	  \LARGE{F}\large{ACULTAD DE } \LARGE{C}\large{IENCIAS}\\[2cm]			% Crea el nombre estilizado de la facultad
	
	  \large{
	Titulo Provisional  }\\[2cm]											% Ingresa el titulo de la tesis
	
	  \huge{
	T \hspace{1cm} E \hspace{1cm} S \hspace{1cm} I \hspace{1cm} S  }\\[1cm]	% Crea la palabra: T E S I S
	
	  \large{QUE PARA OBTENER EL TÍTULO DE:}\\[1cm]
	
	  \large{
	Físico  }\\[1cm]														% Ingresa el titulo a obtener
	
	  \large{PRESENTA:}\\[1cm]
	
	  \large{
	Felipe Alvarez Zuloaga  }\\[1cm]										% Ingresa nombre del estudiante
	
	  \large{
	TUTOR  }\\[1cm]
	
	  \large{
	Fernando García Guillermo Montero  }									% Ingresa el nombre del asesor
	}

\end{minipage}



           														%  Portada
%%%%%%%%%%%%%%%%%%%%%%%%%%%%%%%%%%%%%%%%%%%%%%%
\thispagestyle{empty}															% Evita encabezado y pie de página
\begin{flushright}															% Alinea el texto a la derecha
	\textbf{\textit{Pequeña dedicatoria.}}									% Incluimos la dedicatoria
\end{flushright}
           										% Dedicatoria
%%%%%%%%%%%%%%%%%%%%%%%%%%%%%%%%%%%%%%%%%%%%%%%
\pagestyle{empty}
\include{Indice/Indice_fc}           															% Índice
\pagestyle{fancy}	
%%%%%%%%%%%%%%%%%%%%%%%%%%%%%%%%%%%%%%%%%%%%%%%
%%%%%%%%%%%%%%%%%%%%%%%%%%%%%%%%%%%%%%%%%%%%%%%
% INTRODUCCION   %
%%%%%%%%%%%%%%%%%%%%%%%%%%%%%%%%%%%%%%%%%%%%%%%
\chapter{Introducción y antecedentes}
%%%%%%%%%%%%%%%%%%%%%%%%%%%%%%%%%%%%%%%%%%%%%%%%%%%%%%%%%%%%%%%%%%%%%%%%%
\section{Microfísica de nubes} 
La lluvia es uno de los fenómenos meteorológicos mas importantes y más comunes en la naturaleza. Ésta es fundamental en el ciclo del agua y en muchas otras actividades de origen humano que van desde la agricultura, hasta la prevención de desastres naturales. 
Todas esas actividades dependen finalmente de tomar buenas medidas, hacer buenas estimaciones o  buenas predicciones de diferentes variables, dependiendo de que se quiera hacer, como pueden ser la precipitacion acumulada, intensidad, tiempo de duracion de la lluvia, etc...
Por lo tanto es necesario mejorar nuestro entendimiento acerca de estos fenómenos,


%%%%%%%%%%%%%%%%%%%%%%%%%%%%%%%%%%%%%%%%%%%%%%%%%%%%%%%%%%%%%%%%%%%%%%%%%
%                           Objetivo                                    %
%%%%%%%%%%%%%%%%%%%%%%%%%%%%%%%%%%%%%%%%%%%%%%%%%%%%%%%%%%%%%%%%%%%%%%%%%
\section{Introducción a procesos de lluvia caliente}
EEste trabajo tiene por objetivo ...Este trabajo tiene por objetivo ...Este trabajo tiene por objetivo ...Este trabajo tiene por objetivo ...ste trabajo tiene por objetivo ...
Este trabajo tiene por objetivo ...Este trabajo tiene por objetivo ...Este trabajo tiene por objetivo ...Este trabajo tiene por objetivo ...
Este trabajo tiene por objetivo ...vEste trabajo tiene por objetivo ...Este trabajo tiene por objetivo ...Este trabajo tiene por objetivo ...Este trabajo tiene por objetivo ...
Este trabajo tiene por objetivo ...Este trabajo tiene por objetivo ...Este trabajo tiene por objetivo ...
%%%%%%%%%%%%%%%%%%%%%%%%%%%%%%%%%%%%%%%%%%%%%%%%%%%%%%%%%%%%%%%%%%%%%%%%%
%                           Motivación y estado del arte                %
%%%%%%%%%%%%%%%%%%%%%%%%%%%%%%%%%%%%%%%%%%%%%%%%%%%%%%%%%%%%%%%%%%%%%%%%%
\section{Planteamiento de hipótesis}
Planteamiento de la hipotesis a validar/refutar           												% Introduccion
%%%%%%%%%%%%%%%%%%%%%%%%%%%%%%%%%%%%%%%%%%%%%%%
%%%%%%%%%%%%%%%%%%%%%%%%%%%%%%%%%%%%%%%%%%%%%%%
% METODOLOGIA   %
%%%%%%%%%%%%%%%%%%%%%%%%%%%%%%%%%%%%%%%%%%%%%%%
\chapter{Marco teórico}
%%%%%%%%%%%%%%%%%%%%%%%%%%%%%%%%%%%%%%%%%%%%%%%%%%%%%%%%%%%%%%%%%%%%%%%%%
\section{Regiones de estudio} 
EEste trabajo tiene por objetivo ...Este trabajo tiene por objetivo ...Este trabajo tiene por objetivo ...Este trabajo tiene por objetivo ...ste trabajo tiene por objetivo ...
Este trabajo tiene por objetivo ...Este trabajo tiene por objetivo ...Este trabajo tiene por objetivo ...Este trabajo tiene por objetivo ...
Este trabajo tiene por objetivo ...vEste trabajo tiene por objetivo ...Este trabajo tiene por objetivo ...Este trabajo tiene por objetivo ...Este trabajo tiene por objetivo ...
Este trabajo tiene por objetivo ...Este trabajo tiene por objetivo ...Este trabajo tiene por objetivo ...
EEste trabajo tiene por objetivo ...Este trabajo tiene por objetivo ...Este trabajo tiene por objetivo ...Este trabajo tiene por objetivo ...ste trabajo tiene por objetivo ...
Este trabajo tiene por objetivo ...Este trabajo tiene por objetivo ...Este trabajo tiene por objetivo ...Este trabajo tiene por objetivo ...
Este trabajo tiene por objetivo ...vEste trabajo tiene por objetivo ...Este trabajo tiene por objetivo ...Este trabajo tiene por objetivo ...Este trabajo tiene por objetivo ...
Este trabajo tiene por objetivo ...Este trabajo tiene por objetivo ...Este trabajo tiene por objetivo ...
EEste trabajo tiene por objetivo ...Este trabajo tiene por objetivo ...Este trabajo tiene por objetivo ...Este trabajo tiene por objetivo ...ste trabajo tiene por objetivo ...
Este trabajo tiene por objetivo ...Este trabajo tiene por objetivo ...Este trabajo tiene por objetivo ...Este trabajo tiene por objetivo ...
Este trabajo tiene por objetivo ...vEste trabajo tiene por objetivo ...Este trabajo tiene por objetivo ...Este trabajo tiene por objetivo ...Este trabajo tiene por objetivo ...
Este trabajo tiene por objetivo ...Este trabajo tiene por objetivo ...Este trabajo tiene por objetivo ...
EEste trabajo tiene por objetivo ...Este trabajo tiene por objetivo ...Este trabajo tiene por objetivo ...Este trabajo tiene por objetivo ...ste trabajo tiene por objetivo ...
Este trabajo tiene por objetivo ...Este trabajo tiene por objetivo ...Este trabajo tiene por objetivo ...Este trabajo tiene por objetivo ...
Este trabajo tiene por objetivo ...vEste trabajo tiene por objetivo ...Este trabajo tiene por objetivo ...Este trabajo tiene por objetivo ...Este trabajo tiene por objetivo ...
Este trabajo tiene por objetivo ...Este trabajo tiene por objetivo ...Este trabajo tiene por objetivo ...
EEste trabajo tiene por objetivo ...Este trabajo tiene por objetivo ...Este trabajo tiene por objetivo ...Este trabajo tiene por objetivo ...ste trabajo tiene por objetivo ...
Este trabajo tiene por objetivo ...Este trabajo tiene por objetivo ...Este trabajo tiene por objetivo ...Este trabajo tiene por objetivo ...
Este trabajo tiene por objetivo ...vEste trabajo tiene por objetivo ...Este trabajo tiene por objetivo ...Este trabajo tiene por objetivo ...Este trabajo tiene por objetivo ...
Este trabajo tiene por objetivo ...Este trabajo tiene por objetivo ...Este trabajo tiene por objetivo ...
EEste trabajo tiene por objetivo ...Este trabajo tiene por objetivo ...Este trabajo tiene por objetivo ...Este trabajo tiene por objetivo ...ste trabajo tiene por objetivo ...
Este trabajo tiene por objetivo ...Este trabajo tiene por objetivo ...Este trabajo tiene por objetivo ...Este trabajo tiene por objetivo ...
Este trabajo tiene por objetivo ...vEste trabajo tiene por objetivo ...Este trabajo tiene por objetivo ...Este trabajo tiene por objetivo ...Este trabajo tiene por objetivo ...
Este trabajo tiene por objetivo ...Este trabajo tiene por objetivo ...Este trabajo tiene por objetivo ...
EEste trabajo tiene por objetivo ...Este trabajo tiene por objetivo ...Este trabajo tiene por objetivo ...Este trabajo tiene por objetivo ...ste trabajo tiene por objetivo ...
Este trabajo tiene por objetivo ...Este trabajo tiene por objetivo ...Este trabajo tiene por objetivo ...Este trabajo tiene por objetivo ...
Este trabajo tiene por objetivo ...vEste trabajo tiene por objetivo ...Este trabajo tiene por objetivo ...Este trabajo tiene por objetivo ...Este trabajo tiene por objetivo ...
Este trabajo tiene por objetivo ...Este trabajo tiene por objetivo ...Este trabajo tiene por objetivo ...
EEste trabajo tiene por objetivo ...Este trabajo tiene por objetivo ...Este trabajo tiene por objetivo ...Este trabajo tiene por objetivo ...ste trabajo tiene por objetivo ...
Este trabajo tiene por objetivo ...Este trabajo tiene por objetivo ...Este trabajo tiene por objetivo ...Este trabajo tiene por objetivo ...
Este trabajo tiene por objetivo ...vEste trabajo tiene por objetivo ...Este trabajo tiene por objetivo ...Este trabajo tiene por objetivo ...Este trabajo tiene por objetivo ...
Este trabajo tiene por objetivo ...Este trabajo tiene por objetivo ...Este trabajo tiene por objetivo ...
EEste trabajo tiene por objetivo ...Este trabajo tiene por objetivo ...Este trabajo tiene por objetivo ...Este trabajo tiene por objetivo ...ste trabajo tiene por objetivo ...
Este trabajo tiene por objetivo ...Este trabajo tiene por objetivo ...Este trabajo tiene por objetivo ...Este trabajo tiene por objetivo ...
Este trabajo tiene por objetivo ...vEste trabajo tiene por objetivo ...Este trabajo tiene por objetivo ...Este trabajo tiene por objetivo ...Este trabajo tiene por objetivo ...
Este trabajo tiene por objetivo ...Este trabajo tiene por objetivo ...Este trabajo tiene por objetivo ...
EEste trabajo tiene por objetivo ...Este trabajo tiene por objetivo ...Este trabajo tiene por objetivo ...Este trabajo tiene por objetivo ...ste trabajo tiene por objetivo ...
Este trabajo tiene por objetivo ...Este trabajo tiene por objetivo ...Este trabajo tiene por objetivo ...Este trabajo tiene por objetivo ...
Este trabajo tiene por objetivo ...vEste trabajo tiene por objetivo ...Este trabajo tiene por objetivo ...Este trabajo tiene por objetivo ...Este trabajo tiene por objetivo ...
Este trabajo tiene por objetivo ...Este trabajo tiene por objetivo ...Este trabajo tiene por objetivo ...
EEste trabajo tiene por objetivo ...Este trabajo tiene por objetivo ...Este trabajo tiene por objetivo ...Este trabajo tiene por objetivo ...ste trabajo tiene por objetivo ...
Este trabajo tiene por objetivo ...Este trabajo tiene por objetivo ...Este trabajo tiene por objetivo ...Este trabajo tiene por objetivo ...
Este trabajo tiene por objetivo ...vEste trabajo tiene por objetivo ...Este trabajo tiene por objetivo ...Este trabajo tiene por objetivo ...Este trabajo tiene por objetivo ...
Este trabajo tiene por objetivo ...Este trabajo tiene por objetivo ...Este trabajo tiene por objetivo ...
EEste trabajo tiene por objetivo ...Este trabajo tiene por objetivo ...Este trabajo tiene por objetivo ...Este trabajo tiene por objetivo ...ste trabajo tiene por objetivo ...
Este trabajo tiene por objetivo ...Este trabajo tiene por objetivo ...Este trabajo tiene por objetivo ...Este trabajo tiene por objetivo ...
Este trabajo tiene por objetivo ...vEste trabajo tiene por objetivo ...Este trabajo tiene por objetivo ...Este trabajo tiene por objetivo ...Este trabajo tiene por objetivo ...
Este trabajo tiene por objetivo ...Este trabajo tiene por objetivo ...Este trabajo tiene por objetivo ...


%%%%%%%%%%%%%%%%%%%%%%%%%%%%%%%%%%%%%%%%%%%%%%%%%%%%%%%%%%%%%%%%%%%%%%%%%
%                           Objetivo                                    %
%%%%%%%%%%%%%%%%%%%%%%%%%%%%%%%%%%%%%%%%%%%%%%%%%%%%%%%%%%%%%%%%%%%%%%%%%
\section{Instrumentación}
EEste trabajo tiene por objetivo ...Este trabajo tiene por objetivo ...Este trabajo tiene por objetivo ...Este trabajo tiene por objetivo ...ste trabajo tiene por objetivo ...
Este trabajo tiene por objetivo ...Este trabajo tiene por objetivo ...Este trabajo tiene por objetivo ...Este trabajo tiene por objetivo ...
Este trabajo tiene por objetivo ...vEste trabajo tiene por objetivo ...Este trabajo tiene por objetivo ...Este trabajo tiene por objetivo ...Este trabajo tiene por objetivo ...
Este trabajo tiene por objetivo ...Este trabajo tiene por objetivo ...Este trabajo tiene por objetivo ...
                 									%Marco teórico 
%%%%%%%%%%%%%%%%%%%%%%%%%%%%%%%%%%%%%%%%%%%%%%%
%%%%%%%%%%%%%%%%%%%%%%%%%%%%%%%%%%%%%%%%%%%%%%%
% ANALISIS   %
%%%%%%%%%%%%%%%%%%%%%%%%%%%%%%%%%%%%%%%%%%%%%%%
\chapter{Analisis de datos}
%%%%%%%%%%%%%%%%%%%%%%%%%%%%%%%%%%%%%%%%%%%%%%%%%%%%%%%%%%%%%%%%%%%%%%%%%
\section{Regiones de estudio} 
EEste trabajo tiene por objetivo ...Este trabajo tiene por objetivo ...Este trabajo tiene por objetivo ...Este trabajo tiene por objetivo ...ste trabajo tiene por objetivo ...
Este trabajo tiene por objetivo ...Este trabajo tiene por objetivo ...Este trabajo tiene por objetivo ...Este trabajo tiene por objetivo ...
Este trabajo tiene por objetivo ...vEste trabajo tiene por objetivo ...Este trabajo tiene por objetivo ...Este trabajo tiene por objetivo ...Este trabajo tiene por objetivo ...
Este trabajo tiene por objetivo ...Este trabajo tiene por objetivo ...Este trabajo tiene por objetivo ...
EEste trabajo tiene por objetivo ...Este trabajo tiene por objetivo ...Este trabajo tiene por objetivo ...Este trabajo tiene por objetivo ...ste trabajo tiene por objetivo ...
Este trabajo tiene por objetivo ...Este trabajo tiene por objetivo ...Este trabajo tiene por objetivo ...Este trabajo tiene por objetivo ...
Este trabajo tiene por objetivo ...vEste trabajo tiene por objetivo ...Este trabajo tiene por objetivo ...Este trabajo tiene por objetivo ...Este trabajo tiene por objetivo ...
Este trabajo tiene por objetivo ...Este trabajo tiene por objetivo ...Este trabajo tiene por objetivo ...
EEste trabajo tiene por objetivo ...Este trabajo tiene por objetivo ...Este trabajo tiene por objetivo ...Este trabajo tiene por objetivo ...ste trabajo tiene por objetivo ...
Este trabajo tiene por objetivo ...Este trabajo tiene por objetivo ...Este trabajo tiene por objetivo ...Este trabajo tiene por objetivo ...
Este trabajo tiene por objetivo ...vEste trabajo tiene por objetivo ...Este trabajo tiene por objetivo ...Este trabajo tiene por objetivo ...Este trabajo tiene por objetivo ...
Este trabajo tiene por objetivo ...Este trabajo tiene por objetivo ...Este trabajo tiene por objetivo ...
EEste trabajo tiene por objetivo ...Este trabajo tiene por objetivo ...Este trabajo tiene por objetivo ...Este trabajo tiene por objetivo ...ste trabajo tiene por objetivo ...
Este trabajo tiene por objetivo ...Este trabajo tiene por objetivo ...Este trabajo tiene por objetivo ...Este trabajo tiene por objetivo ...
Este trabajo tiene por objetivo ...vEste trabajo tiene por objetivo ...Este trabajo tiene por objetivo ...Este trabajo tiene por objetivo ...Este trabajo tiene por objetivo ...
Este trabajo tiene por objetivo ...Este trabajo tiene por objetivo ...Este trabajo tiene por objetivo ...
EEste trabajo tiene por objetivo ...Este trabajo tiene por objetivo ...Este trabajo tiene por objetivo ...Este trabajo tiene por objetivo ...ste trabajo tiene por objetivo ...
Este trabajo tiene por objetivo ...Este trabajo tiene por objetivo ...Este trabajo tiene por objetivo ...Este trabajo tiene por objetivo ...
Este trabajo tiene por objetivo ...vEste trabajo tiene por objetivo ...Este trabajo tiene por objetivo ...Este trabajo tiene por objetivo ...Este trabajo tiene por objetivo ...
Este trabajo tiene por objetivo ...Este trabajo tiene por objetivo ...Este trabajo tiene por objetivo ...
EEste trabajo tiene por objetivo ...Este trabajo tiene por objetivo ...Este trabajo tiene por objetivo ...Este trabajo tiene por objetivo ...ste trabajo tiene por objetivo ...
Este trabajo tiene por objetivo ...Este trabajo tiene por objetivo ...Este trabajo tiene por objetivo ...Este trabajo tiene por objetivo ...
Este trabajo tiene por objetivo ...vEste trabajo tiene por objetivo ...Este trabajo tiene por objetivo ...Este trabajo tiene por objetivo ...Este trabajo tiene por objetivo ...
Este trabajo tiene por objetivo ...Este trabajo tiene por objetivo ...Este trabajo tiene por objetivo ...
EEste trabajo tiene por objetivo ...Este trabajo tiene por objetivo ...Este trabajo tiene por objetivo ...Este trabajo tiene por objetivo ...ste trabajo tiene por objetivo ...
Este trabajo tiene por objetivo ...Este trabajo tiene por objetivo ...Este trabajo tiene por objetivo ...Este trabajo tiene por objetivo ...
Este trabajo tiene por objetivo ...vEste trabajo tiene por objetivo ...Este trabajo tiene por objetivo ...Este trabajo tiene por objetivo ...Este trabajo tiene por objetivo ...
Este trabajo tiene por objetivo ...Este trabajo tiene por objetivo ...Este trabajo tiene por objetivo ...
EEste trabajo tiene por objetivo ...Este trabajo tiene por objetivo ...Este trabajo tiene por objetivo ...Este trabajo tiene por objetivo ...ste trabajo tiene por objetivo ...
Este trabajo tiene por objetivo ...Este trabajo tiene por objetivo ...Este trabajo tiene por objetivo ...Este trabajo tiene por objetivo ...
Este trabajo tiene por objetivo ...vEste trabajo tiene por objetivo ...Este trabajo tiene por objetivo ...Este trabajo tiene por objetivo ...Este trabajo tiene por objetivo ...
Este trabajo tiene por objetivo ...Este trabajo tiene por objetivo ...Este trabajo tiene por objetivo ...
EEste trabajo tiene por objetivo ...Este trabajo tiene por objetivo ...Este trabajo tiene por objetivo ...Este trabajo tiene por objetivo ...ste trabajo tiene por objetivo ...
Este trabajo tiene por objetivo ...Este trabajo tiene por objetivo ...Este trabajo tiene por objetivo ...Este trabajo tiene por objetivo ...
Este trabajo tiene por objetivo ...vEste trabajo tiene por objetivo ...Este trabajo tiene por objetivo ...Este trabajo tiene por objetivo ...Este trabajo tiene por objetivo ...
Este trabajo tiene por objetivo ...Este trabajo tiene por objetivo ...Este trabajo tiene por objetivo ...
EEste trabajo tiene por objetivo ...Este trabajo tiene por objetivo ...Este trabajo tiene por objetivo ...Este trabajo tiene por objetivo ...ste trabajo tiene por objetivo ...
Este trabajo tiene por objetivo ...Este trabajo tiene por objetivo ...Este trabajo tiene por objetivo ...Este trabajo tiene por objetivo ...
Este trabajo tiene por objetivo ...vEste trabajo tiene por objetivo ...Este trabajo tiene por objetivo ...Este trabajo tiene por objetivo ...Este trabajo tiene por objetivo ...
Este trabajo tiene por objetivo ...Este trabajo tiene por objetivo ...Este trabajo tiene por objetivo ...
EEste trabajo tiene por objetivo ...Este trabajo tiene por objetivo ...Este trabajo tiene por objetivo ...Este trabajo tiene por objetivo ...ste trabajo tiene por objetivo ...
Este trabajo tiene por objetivo ...Este trabajo tiene por objetivo ...Este trabajo tiene por objetivo ...Este trabajo tiene por objetivo ...
Este trabajo tiene por objetivo ...vEste trabajo tiene por objetivo ...Este trabajo tiene por objetivo ...Este trabajo tiene por objetivo ...Este trabajo tiene por objetivo ...
Este trabajo tiene por objetivo ...Este trabajo tiene por objetivo ...Este trabajo tiene por objetivo ...
EEste trabajo tiene por objetivo ...Este trabajo tiene por objetivo ...Este trabajo tiene por objetivo ...Este trabajo tiene por objetivo ...ste trabajo tiene por objetivo ...
Este trabajo tiene por objetivo ...Este trabajo tiene por objetivo ...Este trabajo tiene por objetivo ...Este trabajo tiene por objetivo ...
Este trabajo tiene por objetivo ...vEste trabajo tiene por objetivo ...Este trabajo tiene por objetivo ...Este trabajo tiene por objetivo ...Este trabajo tiene por objetivo ...
Este trabajo tiene por objetivo ...Este trabajo tiene por objetivo ...Este trabajo tiene por objetivo ...


%%%%%%%%%%%%%%%%%%%%%%%%%%%%%%%%%%%%%%%%%%%%%%%%%%%%%%%%%%%%%%%%%%%%%%%%%
%                           Objetivo                                    %
%%%%%%%%%%%%%%%%%%%%%%%%%%%%%%%%%%%%%%%%%%%%%%%%%%%%%%%%%%%%%%%%%%%%%%%%%
\section{Ajuste de modelos}
EEste trabajo tiene por objetivo ...Este trabajo tiene por objetivo ...Este trabajo tiene por objetivo ...Este trabajo tiene por objetivo ...ste trabajo tiene por objetivo ...
Este trabajo tiene por objetivo ...Este trabajo tiene por objetivo ...Este trabajo tiene por objetivo ...Este trabajo tiene por objetivo ...
Este trabajo tiene por objetivo ...vEste trabajo tiene por objetivo ...Este trabajo tiene por objetivo ...Este trabajo tiene por objetivo ...Este trabajo tiene por objetivo ...
Este trabajo tiene por objetivo ...Este trabajo tiene por objetivo ...Este trabajo tiene por objetivo ...
                 													%Analisis de resultados
%%%%%%%%%%%%%%%%%%%%%%%%%%%%%%%%%%%%%%%%%%%%%%%
%%%%%%%%%%%%%%%%%%%%%%%%%%%%%%%%%%%%%%%%%%%%%%%
% ANALISIS   %
%%%%%%%%%%%%%%%%%%%%%%%%%%%%%%%%%%%%%%%%%%%%%%%
\chapter{Conclusiones}
%%%%%%%%%%%%%%%%%%%%%%%%%%%%%%%%%%%%%%%%%%%%%%%%%%%%%%%%%%%%%%%%%%%%%%%%%
\section{Resultados} 
EEste trabajo tiene por objetivo ...Este trabajo tiene por objetivo ...Este trabajo tiene por objetivo ...Este trabajo tiene por objetivo ...ste trabajo tiene por objetivo ...
Este trabajo tiene por objetivo ...Este trabajo tiene por objetivo ...Este trabajo tiene por objetivo ...Este trabajo tiene por objetivo ...\cite{steiner_peaks_1987}
Este trabajo tiene por objetivo ...vEste trabajo tiene por objetivo ...Este trabajo tiene por objetivo ...Este trabajo tiene por objetivo ...Este trabajo tiene por objetivo ...
Este trabajo tiene por objetivo ...Este trabajo tiene por objetivo ...Este trabajo tiene por objetivo ...
EEste trabajo tiene por objetivo ...Este trabajo tiene por objetivo ...Este trabajo tiene por objetivo ...Este trabajo tiene por objetivo ...ste trabajo tiene por objetivo ...
Este trabajo tiene por objetivo ...Este trabajo tiene por objetivo ...Este trabajo tiene por objetivo ...Este trabajo tiene por objetivo ...
Este trabajo tiene por objetivo ...vEste trabajo tiene por objetivo ...Este trabajo tiene por objetivo ...Este trabajo tiene por objetivo ...Este trabajo tiene por objetivo ...
Este trabajo tiene por objetivo ...Este trabajo tiene por objetivo ...Este trabajo tiene por objetivo ...
EEste trabajo tiene por objetivo ...Este trabajo tiene por objetivo ...Este trabajo tiene por objetivo ...Este trabajo tiene por objetivo ...ste trabajo tiene por objetivo ...
Este trabajo tiene por objetivo ...Este trabajo tiene por objetivo ...Este trabajo tiene por objetivo ...Este trabajo tiene por objetivo ...
Este trabajo tiene por objetivo ...vEste trabajo tiene por objetivo ...Este trabajo tiene por objetivo ...Este trabajo tiene por objetivo ...Este trabajo tiene por objetivo ...
Este trabajo tiene por objetivo ...Este trabajo tiene por objetivo ...Este trabajo tiene por objetivo ...
EEste trabajo tiene por objetivo ...Este trabajo tiene por objetivo ...Este trabajo tiene por objetivo ...Este trabajo tiene por objetivo ...ste trabajo tiene por objetivo ...
Este trabajo tiene por objetivo ...Este trabajo tiene por objetivo ...Este trabajo tiene por objetivo ...Este trabajo tiene por objetivo ...
Este trabajo tiene por objetivo ...vEste trabajo tiene por objetivo ...Este trabajo tiene por objetivo ...Este trabajo tiene por objetivo ...Este trabajo tiene por objetivo ...
Este trabajo tiene por objetivo ...Este trabajo tiene por objetivo ...Este trabajo tiene por objetivo ...
EEste trabajo tiene por objetivo ...Este trabajo tiene por objetivo ...Este trabajo tiene por objetivo ...Este trabajo tiene por objetivo ...ste trabajo tiene por objetivo ...
Este trabajo tiene por objetivo ...Este trabajo tiene por objetivo ...Este trabajo tiene por objetivo ...Este trabajo tiene por objetivo ...
Este trabajo tiene por objetivo ...vEste trabajo tiene por objetivo ...Este trabajo tiene por objetivo ...Este trabajo tiene por objetivo ...Este trabajo tiene por objetivo ...
Este trabajo tiene por objetivo ...Este trabajo tiene por objetivo ...Este trabajo tiene por objetivo ...
EEste trabajo tiene por objetivo ...Este trabajo tiene por objetivo ...Este trabajo tiene por objetivo ...Este trabajo tiene por objetivo ...ste trabajo tiene por objetivo ...
Este trabajo tiene por objetivo ...Este trabajo tiene por objetivo ...Este trabajo tiene por objetivo ...Este trabajo tiene por objetivo ...
Este trabajo tiene por objetivo ...vEste trabajo tiene por objetivo ...Este trabajo tiene por objetivo ...Este trabajo tiene por objetivo ...Este trabajo tiene por objetivo ...
Este trabajo tiene por objetivo ...Este trabajo tiene por objetivo ...Este trabajo tiene por objetivo ...
EEste trabajo tiene por objetivo ...Este trabajo tiene por objetivo ...Este trabajo tiene por objetivo ...Este trabajo tiene por objetivo ...ste trabajo tiene por objetivo ...
Este trabajo tiene por objetivo ...Este trabajo tiene por objetivo ...Este trabajo tiene por objetivo ...Este trabajo tiene por objetivo ...
Este trabajo tiene por objetivo ...vEste trabajo tiene por objetivo ...Este trabajo tiene por objetivo ...Este trabajo tiene por objetivo ...Este trabajo tiene por objetivo ...
Este trabajo tiene por objetivo ...Este trabajo tiene por objetivo ...Este trabajo tiene por objetivo ...
EEste trabajo tiene por objetivo ...Este trabajo tiene por objetivo ...Este trabajo tiene por objetivo ...Este trabajo tiene por objetivo ...ste trabajo tiene por objetivo ...
Este trabajo tiene por objetivo ...Este trabajo tiene por objetivo ...Este trabajo tiene por objetivo ...Este trabajo tiene por objetivo ...
Este trabajo tiene por objetivo ...vEste trabajo tiene por objetivo ...Este trabajo tiene por objetivo ...Este trabajo tiene por objetivo ...Este trabajo tiene por objetivo ...
Este trabajo tiene por objetivo ...Este trabajo tiene por objetivo ...Este trabajo tiene por objetivo ...
EEste trabajo tiene por objetivo ...Este trabajo tiene por objetivo ...Este trabajo tiene por objetivo ...Este trabajo tiene por objetivo ...ste trabajo tiene por objetivo ...
Este trabajo tiene por objetivo ...Este trabajo tiene por objetivo ...Este trabajo tiene por objetivo ...Este trabajo tiene por objetivo ...
Este trabajo tiene por objetivo ...vEste trabajo tiene por objetivo ...Este trabajo tiene por objetivo ...Este trabajo tiene por objetivo ...Este trabajo tiene por objetivo ...
Este trabajo tiene por objetivo ...Este trabajo tiene por objetivo ...Este trabajo tiene por objetivo ...
EEste trabajo tiene por objetivo ...Este trabajo tiene por objetivo ...Este trabajo tiene por objetivo ...Este trabajo tiene por objetivo ...ste trabajo tiene por objetivo ...
Este trabajo tiene por objetivo ...Este trabajo tiene por objetivo ...Este trabajo tiene por objetivo ...Este trabajo tiene por objetivo ...
Este trabajo tiene por objetivo ...vEste trabajo tiene por objetivo ...Este trabajo tiene por objetivo ...Este trabajo tiene por objetivo ...Este trabajo tiene por objetivo ...
Este trabajo tiene por objetivo ...Este trabajo tiene por objetivo ...Este trabajo tiene por objetivo ...
EEste trabajo tiene por objetivo ...Este trabajo tiene por objetivo ...Este trabajo tiene por objetivo ...Este trabajo tiene por objetivo ...ste trabajo tiene por objetivo ...
Este trabajo tiene por objetivo ...Este trabajo tiene por objetivo ...Este trabajo tiene por objetivo ...Este trabajo tiene por objetivo ...
Este trabajo tiene por objetivo ...vEste trabajo tiene por objetivo ...Este trabajo tiene por objetivo ...Este trabajo tiene por objetivo ...Este trabajo tiene por objetivo ...
Este trabajo tiene por objetivo ...Este trabajo tiene por objetivo ...Este trabajo tiene por objetivo ...
EEste trabajo tiene por objetivo ...Este trabajo tiene por objetivo ...Este trabajo tiene por objetivo ...Este trabajo tiene por objetivo ...ste trabajo tiene por objetivo ...
Este trabajo tiene por objetivo ...Este trabajo tiene por objetivo ...Este trabajo tiene por objetivo ...Este trabajo tiene por objetivo ...
Este trabajo tiene por objetivo ...vEste trabajo tiene por objetivo ...Este trabajo tiene por objetivo ...Este trabajo tiene por objetivo ...Este trabajo tiene por objetivo ...
Este trabajo tiene por objetivo ...Este trabajo tiene por objetivo ...Este trabajo tiene por objetivo ...


%%%%%%%%%%%%%%%%%%%%%%%%%%%%%%%%%%%%%%%%%%%%%%%%%%%%%%%%%%%%%%%%%%%%%%%%%
%                           Objetivo                                    %
%%%%%%%%%%%%%%%%%%%%%%%%%%%%%%%%%%%%%%%%%%%%%%%%%%%%%%%%%%%%%%%%%%%%%%%%%
\section{Trabajo futuro}
EEste trabajo tiene por objetivo ...Este trabajo tiene por objetivo ...Este trabajo tiene por objetivo ...Este trabajo tiene por objetivo ...ste trabajo tiene por objetivo ...
Este trabajo tiene por objetivo ...Este trabajo tiene por objetivo ...Este trabajo tiene por objetivo ...Este trabajo tiene por objetivo ...
Este trabajo tiene por objetivo ...vEste trabajo tiene por objetivo ...Este trabajo tiene por objetivo ...Este trabajo tiene por objetivo ...Este trabajo tiene por objetivo ...
Este trabajo tiene por objetivo ...Este trabajo tiene por objetivo ...Este trabajo tiene por objetivo ...
                 										%Analisis de resultados
%%%%%%%%%%%%%%%%%%%%%%%%%%%%%%%%%%%%%%%%%%%%%%%
\addcontentsline{toc}{chapter}{Referencias}
\bibliographystyle{plain}
\bibliography{Bibliografia/bib} 

\end{document}

